\section{Design of a robust and interactive GUI}
The current MusEEG GUI is rather clunky and unintuitive. Redesigning the GUI with the goal of giving the user more interactive control over the MIDI patterns being played will result in more musical freedom while using the MusEEG. 

For example, using an 8-bar piano roll MIDI sequencer as an alternative to simple chord and chord arpeggios would provide the user with more musical options to choose from, as well as a more interactive overall experience. 

\section{A SuperCollider Pattern Designer}
Currently, the only way to design SuperCollider patterns beyond the ones provided in the /SuperCollider directory is through programming them in the language itself. Because one of MusEEG's goals is to provide a highly accessible musical interface, a more intuitive way of designing MusEEG-controlled SuperCollider patterns must be implemented. 


\section{A Hybrid EEG Classification System}
Success has been found in building hybrid BCI systems that implement both facial expression recognition and motor imagery simultaneously\cite{gsu:facialexpression}. Since the classification methods used in motor imagery and facial expression recognition are similar, methods to implement a hybrid motor imagery-facial expression recognition BCMI will be added. The addition of motor imagery commands to the BCMI will increase the total number of available commands to the user, allowing for greater variety in performance without having to redefine the command-MIDI dictionary in the system.
